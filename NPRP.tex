\documentclass[final,letterpaper,oneside,12pt]{article}
\usepackage{nasapostdoc}
\usepackage{graphicx}
\usepackage{subcaption}
\usepackage[plainpages=false,colorlinks=true,urlcolor=black,citecolor=blue,linkcolor=blue,urlcolor=blue]{hyperref}
\usepackage{cleveref}
\usepackage{natbib}

\title{Toward Robust Autonomous Exposure Control: Developing a Data Pipeline for UV Astrophysics Missions}
\author{Anna Childs}

\project{0322-NPP-NOV25-JPL-Astrophys: Autonomous Space Observatories for Astrophysics} 
\institution{Jet Propulsion Laboratory, California Institute of Technology}  
\advisor{[Prospective NASA Advisor Name]} 

\begin{document}
\maketitle

\begin{abstract}
Autonomous exposure control has recently emerged as a transformative technology in astrophysics CubeSats, pioneered by the Star-Planet Activity Research CubeSat (SPARCS). However, the technology has never been implemented in a general astrophysics context, and its performance under highly variable flare conditions remains uncertain. This proposal outlines a postdoctoral research program to design the initial data pipeline for such systems, with emphasis on testing, validating, and refining the onboard dynamic exposure algorithm and its interface with ground-based reduction. By integrating flare simulations, detector models, and pipeline development, I will establish the foundation for robust UV time-domain astronomy in current and future small missions, reducing risk and enabling confident interpretation of flare-driven astrophysical variability.
\end{abstract}

\section{Statement of the Problem}
Ultraviolet monitoring of low-mass stars is limited by flare-driven variability that can saturate detectors by orders of magnitude above quiescent flux \citep{Loyd2018,MacGregor2021}. The SPARCS mission is the first to incorporate \emph{autonomous dynamic exposure control} to mitigate saturation \citep{Ramiaramanantsoa2021}. Yet uncertainties remain: (1) how well algorithms generalize across flare morphologies, (2) how saturated frames bias flare statistics, and (3) how to integrate on-orbit autonomous control with a ground pipeline for science delivery. A rigorous data pipeline is urgently needed.

\section{Science Background}
Past small missions (MOST, BRITE, ASTERIA) probed stellar variability in the optical. SPARCS extends this into the UV, advancing technology with delta-doped CCDs and detector-integrated filters \citep{Shkolnik2025}. Its novelty is real-time exposure adaptation, based on point-spread maxima rather than pixel-count thresholds. This represents a paradigm shift from solar observatories (Yohkoh, TRACE, Hinode) where algorithms targeted extended sources. Building a pipeline around this new mode of observation is central to extracting reliable flare frequency distributions and rotational modulation signals, and will directly support NASA’s exoplanet habitability science.

\section{Objectives and Expected Significance}
\begin{itemize}
    \item Deliver an initial data pipeline capable of handling autonomous exposure datasets, including frame-by-frame metadata, saturation events, and gain changes.
    \item Quantify uncertainties in flare recovery given time-delay feedback in the control loop.
    \item Generalize lessons from SPARCS to future JPL-led UV missions, informing detector and software design.
    \item Train a methodology for handling sparse/heterogeneous datasets, anticipating integration with JWST and Habitable Worlds Observatory UV precursors.
\end{itemize}

\section{General Methodology}
I will:
\begin{enumerate}
    \item Use synthetic flare light curves and detector simulations \citep{Ramiaramanantsoa2021} to test autonomous control scenarios across flare amplitudes and cadences.
    \item Develop a Python-based ground pipeline (building on ALICE \citep{Ramiaramanantsoa2025}) to ingest raw frames, apply dark/flat corrections, detect saturation/blooming, and recover photometry.
    \item Validate against archival GALEX and HST flare statistics, testing consistency of recovered flare frequency distributions.
    \item Incorporate machine-learning based anomaly detection for saturated frames, providing error models rather than simple flagging.
\end{enumerate}

\section{Explanation of New Techniques}
The novelty is two-fold:
\begin{itemize}
    \item \textbf{Onboard autonomy:} exposure adjustments occur frame-to-frame, a first in astrophysics CubeSats.
    \item \textbf{Pipeline integration:} unlike fixed-exposure data, these datasets are inherently heterogeneous. I will design reduction methods that propagate exposure-time uncertainty into light-curve analyses, ensuring flare energies are unbiased.
\end{itemize}

\section{Perceived Impact to State of Knowledge}
This work directly addresses gaps in UV flare frequency statistics and detector performance, both critical for exoplanet habitability studies \citep{Segura2010,Luger2015}. The pipeline will enable SPARCS and follow-on missions to deliver trustworthy science products, extending NASA’s time-domain UV capabilities and de-risking autonomous technologies for larger observatories.

\section{Work Plan}
\begin{itemize}
    \item \textbf{Year 1:} Compile flare simulations and detector models; prototype ingestion and calibration modules.
    \item \textbf{Year 2:} Implement autonomous-exposure-aware photometry and flare recovery testing; quantify uncertainties under saturation.
    \item \textbf{Year 3:} Generalize pipeline for future missions; release open-source version with documentation; publish validation studies.
\end{itemize}

\newpage
\singlespacing
\setlength{\bibsep}{0ex}
\bibliographystyle{abbrvnat}
\bibliography{nasa-bib-nprp}

\end{document}