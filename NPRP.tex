\documentclass[final,letterpaper,oneside,12pt]{article}
\usepackage{nasapostdoc}
\usepackage{graphicx}
\usepackage{subcaption}
\usepackage[plainpages=false,colorlinks=true,urlcolor=black,citecolor=blue,linkcolor=blue,urlcolor=blue]{hyperref}
\usepackage{cleveref}
\usepackage{natbib}

\title{Characterizing Stellar Flare Frequency Distributions with SPARCS: Toward Robust Habitability Assessments}
\author{Anna Childs}

\project{0322-NPP-NOV25-JPL-Astrophys: Autonomous Space Observatories for Astrophysics} 
\institution{Jet Propulsion Laboratory, California Institute of Technology}  
\advisor{[Prospective NASA Advisor Name]} 

\begin{document}
\maketitle

\begin{abstract}
Ultraviolet flares from low-mass stars strongly influence the atmospheres and habitability of orbiting planets, but flare frequency distributions (FFDs) remain poorly constrained in the UV. The Star-Planet Activity Research CubeSat (SPARCS) is the first mission designed to deliver long-baseline, high-cadence UV monitoring of M and K stars. However, its novel use of autonomous exposure control introduces new complexities for data analysis. I propose to develop a robust data pipeline to characterize stellar FFDs from SPARCS, quantifying uncertainties introduced by dynamic exposure adjustments and saturation, and integrating the resulting distributions into models of planetary atmospheric loss and habitability. This work will both enable the core SPARCS science goals and provide critical flare statistics for assessing exoplanet habitability across low-mass stars.
\end{abstract}

\section{Statement of the Problem}
Low-mass stars dominate the exoplanet population accessible to current and near-future missions. Their frequent UV flares can erode atmospheres, drive abiotic oxygen buildup, and bias biosignature interpretations \citep{Segura2010,Luger2015}. While optical flare statistics are well studied, UV FFDs remain sparse due to limited time-domain coverage \citep{Loyd2018,France2020}. SPARCS is the first CubeSat dedicated to long-duration UV monitoring of K and M stars, uniquely positioned to capture the rare, high-energy flares most relevant to habitability \citep{Shkolnik2025}. Yet flare recovery is complicated by SPARCS’ onboard \emph{autonomous dynamic exposure control}, which adjusts integration times in near-real time to mitigate detector saturation during flares \citep{Ramiaramanantsoa2021}. A dedicated pipeline is required to extract flare statistics robustly from such heterogeneous data.

\section{Science Background}
Flare frequency distributions (FFDs) describe the occurrence rate of flares as a function of energy. Extrapolating FFDs to high energies provides essential inputs for atmospheric escape and photochemistry models. Current FFDs for low-mass stars in the UV are limited to archival HST snapshots and GALEX serendipitous detections, typically sampling minutes per star. SPARCS extends this by three orders of magnitude in dwell time, enabling statistically robust FFDs spanning quiescent through extreme flare states. Accurate recovery of flare energies, however, requires accounting for frame-to-frame exposure changes and occasional saturation in the dynamic control system.

\section{Objectives and Expected Significance}
\begin{itemize}
    \item Develop a data pipeline optimized for flare recovery from SPARCS’ dynamically exposed datasets.
    \item Derive flare frequency distributions for a representative set of K and M dwarfs across ages and activity levels.
    \item Quantify the impact of dynamic exposure and saturation on flare energy uncertainties.
    \item Implement resulting FFDs into atmospheric escape and photochemistry models, constraining habitability outcomes for exoplanets around low-mass stars.
\end{itemize}

\section{General Methodology}
\begin{enumerate}
    \item \textbf{Simulated testing:} Use synthetic flare light curves and detector models to test flare recovery under dynamic exposure control, validating against known flare templates.
    \item \textbf{Pipeline development:} Extend the ALICE framework \citep{Ramiaramanantsoa2025} to ingest heterogeneous SPARCS datasets, correct for exposure-time changes, identify blooming events, and recover flare energies.
    \item \textbf{FFD construction:} Apply Bayesian inference to derive FFDs for SPARCS targets, propagating exposure uncertainties into flare energy distributions.
    \item \textbf{Habitability integration:} Couple FFD results to atmospheric escape and photochemical models \citep{Segura2010,Luger2015} to quantify cumulative UV impacts on exoplanetary environments.
\end{enumerate}

\section{Explanation of New Techniques}
Two innovations drive this project:
\begin{itemize}
    \item \textbf{Autonomous exposure-aware flare recovery:} Incorporating frame-by-frame exposure adjustments into light-curve analysis, ensuring flare energies are unbiased by onboard control.
    \item \textbf{FFD-to-habitability pipeline:} Linking empirical FFDs directly to atmospheric models, enabling SPARCS data to inform habitability assessments in a statistically rigorous way.
\end{itemize}

\section{Perceived Impact to State of Knowledge}
This work will yield the first statistically robust UV FFDs for low-mass stars, advancing our understanding of stellar activity across evolutionary stages. By integrating these FFDs into habitability models, the project will directly inform interpretations of biosignatures and atmospheric escape for planets orbiting K and M dwarfs. Beyond SPARCS, the methodology will generalize to future UV missions, ensuring that autonomous exposure datasets can be translated into reliable astrophysical and astrobiological insights.

\section{Work Plan}
\begin{itemize}
    \item \textbf{Year 1:} Develop flare recovery modules; test with simulated flare light curves and synthetic SPARCS data.
    \item \textbf{Year 2:} Apply pipeline to initial SPARCS observations; derive preliminary FFDs for benchmark targets (e.g., AU Mic, AD Leo).
    \item \textbf{Year 3:} Integrate FFDs into atmospheric escape/photochemistry models; release open-source pipeline and publish results.
\end{itemize}

\newpage
\singlespacing
\setlength{\bibsep}{0ex}
\bibliographystyle{abbrvnat}
\bibliography{nasa-bib-nprp}

\end{document}